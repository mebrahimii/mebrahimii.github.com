\documentclass{letter}

\usepackage{csquotes}
\usepackage[margin=1in]{geometry}

\newcommand{\heading}[1]{{\large \textsc{#1}}}

\begin{document}

\heading{COMP 282 - Advanced Data Structures (Fall, 2018)}

\begin{itemize}
  \item[] {\bf Professor:} Adam Clark (adam.clark@csun.edu)
  \item[] {\bf Section:} 16819
  \item[] {\bf Lectures:} JD2221, We 1900-2145
  \item[] {\bf Book:} Data Abstraction and Problem Solving with Java
    (J. Prichard \& F. Carrano)
  \item[] {\bf Office Hours:} JD3338 We 1745-1845
  \item[] {\bf Prerequisites:} COMP 182/L, MATH 150A
\end{itemize}

\heading{Objectives}

Throughout this course, the student will become familiar with data structures
and algorithms used to efficiently manage large volumes of information.  In
particular, the student will become comfortable implementing and using hash
tables and trees.  Sorting and searching of large volumes of information will
also be discussed, as well as their representation in persistent memory
structures, such as indexed files.  Finally, the student will be introduced to
formalized systems for data storage: databases.

\heading{Schedule}

\begin{center}
  \begin{tabular}{ | l | l | l | }
    \hline
    {\bf Wednesday} & {\bf Deadlines} & \\
    \hline
    {\bf August 29} & & Class Introduction, Data Structures Review \\
    {\bf September 5} & & Balance and Tree Rotations \\
    {\bf September 12} & & AVL Trees, 2-3-4 Trees \\
    {\bf September 19} & & Red-Black Trees \\
    {\bf September 26} & Project 1 & Introduction to Graphs \\
    {\bf October 3} & & Searching Graphs (BFS and DFS) \\
    {\bf October 10} & & Shortest Path, MST, and Review \\
    {\bf October 17} & Midterm 1 & Introduction to Hashing \\
    {\bf October 24} & & Hashing and Hash Tables \\
    {\bf October 31} & & Hashing Algorithms, Table Collision \\
    {\bf November 7} & Project 2 & Files, Data Access, and Review \\
    {\bf November 14} & Midterm 2 & Introduction to Databases \\
    {\bf November 21} & & Data Design, Keys, and Normal Forms \\
    {\bf November 28} & & Introduction to SQL \\
    {\bf December 5} & Project 3 & Comprehensive Review \\
    {\bf December 12} & Final & \\
    \hline
  \end{tabular}
\end{center}

\heading{Grading}

\begin{itemize}
  \item[] {\bf Projects:} 30\% (3 Projects each worth 10\%)
  \item[] {\bf Midterms:} 30\% (2 Midterms each worth 15\%)
  \item[] {\bf Final:} 40\% (Comprehensive)
\end{itemize}

Projects will give the student a chance to practice the concepts introduced
during lecture.  The student is expected to turn in a project that will be
testable via an automated process.  The professor will provide a framework
within which the student must complete the assigned task.  If a project is
turned in that does not follow instructions sufficiently to allow for automated
testing, that project will receive no credit.

Midterms are not comprehensive, however the final will be comprehensive.  It
should be noted, however, that the final will place more emphasis on material
covered since the last midterm.  Some time during lecture prior to each exam
will be devoted to a review of material covered on the exam.  In the case of
the final, the entire lecture period will be devoted to a comprehensive review
of the course material.  The student is expected to come to these review
sessions with any questions they would like covered, there will be no
structured review material.

\begin{center}
  \begin{tabular}{ l | l }
    {\bf Score} & {\bf Grade} \\
    \hline
    90 - 100 & A \\
    80 - 89 & B \\
    70 - 79 & C \\
    60 - 69 & D \\
    0 - 59 & F \\
  \end{tabular}
\end{center}

\heading{Academic Dishonesty}

According to CSUN academic policies:

\begin{displayquote}
The maintenance of academic integrity and quality education is the
responsibility of each student within this University and the CSU system.
Cheating or plagiarism in connection with an academic program at a CSU campus
is listed in Section 41301, Title 5, California Code of Regulations as an
offense for which a student may be expelled, suspended or given a less severe
disciplinary sanction. Academic dishonesty is an especially serious offense and
diminishes the quality of scholarship and defrauds those who depend on the
integrity of the University's programs.
\end{displayquote}

All instances of academic dishonesty will be reported to the office of student
affairs.  In addition, the offending assignment or exam will at minimum receive
no credit towards a final grade.  In most cases, the student will simply
receive a failing grade for the course.  If you are unsure as to what
constitutes academic dishonesty, please see the instructor for clarification.

\heading{Attendance}

Students are expected to attend lectures.  While there is no credit given for
attendance, students who miss a session are expected to learn the material on
their own.  Office hours are a student's primary opportunity to ask questions
about course materials and receive help with projects.

\end{document}

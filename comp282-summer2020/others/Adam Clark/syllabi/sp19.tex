\documentclass{letter}

\usepackage{csquotes}
\usepackage[margin=1in]{geometry}

\newcommand{\heading}[1]{{\large \textsc{#1}}}

\begin{document}

\heading{COMP 282 - Advanced Data Structures (Spring, 2019)}

\begin{itemize}
  \item[] {\bf Professor:} Adam Clark (adam.clark@csun.edu)
  \item[] {\bf Section:} 21126
  \item[] {\bf Lectures:} JD1618, Tu 1900-2145
  \item[] {\bf Office Hours:} JD3338 Tu 1745-1845
  \item[] {\bf Prerequisites:} COMP 182/L, MATH 150A
\end{itemize}

\heading{Objectives}

Throughout this course, the student will become familiar with data structures
and algorithms used to efficiently manage large volumes of information.  In
particular, the student will become comfortable implementing and using hash
tables and trees.  Sorting and searching of large volumes of information will
also be discussed, as well as their representation in persistent memory
structures, such as indexed files.  Finally, the student will be introduced to
formalized systems for data storage: databases.

\heading{Schedule}

\begin{center}
  \begin{tabular}{ | l | l | l | }
    \hline
    {\bf Tuesday} & {\bf Exams} & \\
    \hline
    {\bf January 22} & & Class Introduction, Data Structures Review \\
    {\bf January 29} & & Introduction to Graphs \\
    {\bf February 5} & & Searching Graphs (BFS and DFS) \\
    {\bf February 12} & & Shortest Path, Trees, and MST \\
    {\bf February 19} & & Binary Trees, Balance, Rotation, and Review \\
    {\bf February 26} & Midterm 1 & Introduction to Specialized Binary Trees \\
    {\bf March 5} & & AVL Trees and B-Trees \\
    {\bf March 12} & & Red Black Trees \\
    {\bf March 19} & & Spring Break \\
    {\bf March 26} & & Hashing Functions, Hash Tables, and Collisons \\
    {\bf April 2} & & No Class \\
    {\bf April 9} & & Files, Data Access, and Review \\
    {\bf April 16} & Midterm 2 & Introduction to Databases \\
    {\bf April 23} & & Data Design, Keys, and Normal Forms \\
    {\bf April 30} & & Introduction to SQL \\
    {\bf May 7} & & Comprehensive Review \\
    {\bf May 14} & Final & \\
    \hline
  \end{tabular}
\end{center}

\heading{Grading}

\begin{itemize}
  \item[] {\bf Quizzes:} 20\% (4 Quizes each worth 5\%)
  \item[] {\bf Midterms:} 40\% (2 Midterms each worth 20\%)
  \item[] {\bf Final:} 40\% (Comprehensive)
\end{itemize}

Quizzes have no set dates, there are four of them that may be given at any
time.  They are not comprehensive; rather, they are a review of the material
covered since the last quiz or midterm.  These will be multiple choice and will
be delivered at the beginning of lecture.  Quizzes may only be made up at the
discretion of the professor, after explicit documentation as to the reason for
your absence is presented.

Midterms are not comprehensive, however the final will be comprehensive.  These
exams are not multiple choice, expect a mix of short-essay and reproducing
algorithms.  Some time during lecture prior to each exam will be devoted to a
review of material covered on the exam.  In the case of the final, the entire
lecture period will be devoted to a comprehensive review of the course
material.  The student is expected to come to these review sessions with any
questions they would like covered, there will be no structured review material.

\begin{center}
  \begin{tabular}{ l | l }
    {\bf Score} & {\bf Grade} \\
    \hline
    90 - 100 & A \\
    80 - 89 & B \\
    70 - 79 & C \\
    60 - 69 & D \\
    0 - 59 & F \\
  \end{tabular}
\end{center}

\heading{Academic Dishonesty}

According to CSUN academic policies:

\begin{displayquote}
The maintenance of academic integrity and quality education is the
responsibility of each student within this University and the CSU system.
Cheating or plagiarism in connection with an academic program at a CSU campus
is listed in Section 41301, Title 5, California Code of Regulations as an
offense for which a student may be expelled, suspended or given a less severe
disciplinary sanction. Academic dishonesty is an especially serious offense and
diminishes the quality of scholarship and defrauds those who depend on the
integrity of the University's programs.
\end{displayquote}

All instances of academic dishonesty will be reported to the office of student
affairs.  In addition, the offending assignment or exam will at minimum receive
no credit towards a final grade.  In most cases, the student will simply
receive a failing grade for the course.  If you are unsure as to what
constitutes academic dishonesty, please see the instructor for clarification.

\heading{Attendance}

Students are expected to attend lectures.  While there is no credit given for
attendance, students who miss a session are expected to learn the material on
their own.  Office hours are a student's primary opportunity to ask questions
about course materials.

\end{document}
